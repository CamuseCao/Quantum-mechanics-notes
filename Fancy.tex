% !TEX program = xelatex
% !TeX TXS-program:compile = txs:///latexmk/{}[-xelatex -synctex=1 -interaction=nonstopmode -silent -outdir=Temp %.tex]
%%!TEX root = 
%\documentclass[bibstyle=super,font=advance]{Muse-Book}
\documentclass[bibstyle=super,font=empty]{Muse-Book}
\usepackage{lipsum,zhlipsum}
\usepackage{physics}
\begin{document}
\frontmatter
\musecover{title=高等量子力学笔记,subtitle=Created by Camuse \today}
%\musecover{%
%	image=img/cover,
%	logo=img/logo,
%	title=你的题目,
%	subtitle=你的副标题
%}
 
\tableofcontents

\mainmatter

%\part{唯一一卷}
% !TEX root = ./Note.tex
\chapter{量子力学的基本原理与数学表达}

\section{量子力学的基本假设}
\begin{enumerate}
    \item 微观态的状态用希尔伯特空间的波函数表示
    \item 微观系统的力学量用希尔伯特空间的线性厄米算符来表示;本征值方程 \( \hat{A}\ket{a_i}=a_i\ket{a_i} \) \begin{itemize}
        \item 力学量的取值仅限于它的本征值
        \item 完备性: \( \ket*{\psi(t)}=\sum\limits_{i}^{}c_i\ket*{a_i} \) , \( \qty|c_i|^2 \) 给出测量 \( \hat{A} \) 得到 \( a_i \) 的概率 \( \sum\limits_{i}^{}\qty|c_i|^2=1 \) 
        \item 设测量 \( \hat{A} \) 得到 \( a_i \) ,则系统测量后会进入本征态 \( a_i \) 中
    \end{itemize}
    \item 微观系统中每个粒子在直角坐标中的坐标算符和动量算符满足下列关系: \( \qty[\hat{x},\hat{p}]=i\hbar \quad \qty[\hat{x_i},\hat{p_j}]=\delta_{ij}i\hbar \) 。对于同一粒子: \( \qty[r_i,r_j]=0 \quad \qty[p_i,p_j]=0 \) ,不同粒子全部对易均为零。
    \item 微观系统状态 \( \ket{\psi(t)} \) 随时间的变化规律有薛定谔方程来决定。
    \[
        i\hbar\pderivative{t}\ket{\psi(t)}=\hat{H}\ket{\psi(t)}
    \]
    \item 全同性原理 任意两粒子的波函数状态对调后的波函数是对称的(玻色子)或反对称的(费米子)
\end{enumerate}

\section{Dirac 符号,若干常用算符}
\subsection{右矢与左矢(Dirac 记号)}

\begin{enumerate}
    \item 右矢(ket) 态矢量: \( \ket*{\psi(t)},\ket*{\varphi(t)} \) ;本征态: \( \ket*{a_i}.\ket*{x},\ket*{\vec{p}},\ket*{lm} \) 
    \item 左矢: \( \bra{\psi(t)},\bra{\varphi(t)} \) ,本征态: \( \bra{a_i},\bra{s},\bra{\vec{p}}.\bra{lm} \) 
\end{enumerate}

\subsection{内积(inner product)}
\[
    \bra{\varphi(t)}\cdot\ket*{\psi(t)}\equiv\braket*{\varphi(t)}{\psi(t)}
\]
\[
    \bra{a_i}\cdot\ket*{a_j}=\braket*{a_i}{a_j}
\]
内积的性质: \( \braket*{\varphi}{\psi}\equiv\braket*{\psi }{\varphi}^* \) (复共轭)

上式中,令 \( \varphi=\psi  \) ,则有 \( \braket*{\psi }{\psi}\equiv\braket*{\psi }{\psi }^*\Rightarrow\braket*{\psi }{\psi } \) 为实常数。写作 \( \braket*{\psi }{\psi }\equiv\qty|\psi| \) (模方)。 \( \qty|\psi|^2\ge 0 \) ,仅当 \( \ket*{\psi }=0 \) 时,等号成立。

\begin{itemize}
    \item  \( \braket*{\varphi}{\psi }=0,(\ket*{\varphi}\ne 0,\ket{\psi }\ne 0) \)
    \item 态的归一化 \( \braket*{\psi}{\psi }=1 \quad\braket*{a_i}{a_i}=1\) 
    \item Chwartz 不等式 \[
        \qty|\braket*{\varphi}{\psi }|^2\le\braket*{\varphi}{\varphi}\braket{\psi }\]
        \item 三角不等式:  \( \sqrt{(\bra{\varphi }^\dagger\ket*{\psi}(\ket*{\varphi}^\dagger\ket*{\psi }))}\le\sqrt{\braket*{\varphi}{\varphi}}+\sqrt{\braket*{\psi }{\psi }} \) 
\end{itemize}

\subsection{外积(outer product)}
 \( \ketbra*{\varphi}{\psi }\text{or}(\ketbra*{\psi }{\varphi}) \) 外积是算符。
 \[
    \begin{cases}
     (\ketbra*{\varphi}{\psi })\ket*{a}=\braket*{\psi }{a}\ket*{\varphi}\\ 
     \bra{b}(\ketbra*{\varphi}{\psi })=\braket*{b}{\varphi}\bra{\psi }
    \end{cases}
 \]
投影算符  \( \hat{p }_\phi=\ketbra*{\phi }{\phi } \) 
\[
\begin{cases}
    \hat{p}_\phi=\braket*{\phi }{a}\ket*{\phi }\\ 
    \bra{b}\hat{p }_\phi=\braket*{b}{\phi }\bra{\phi }
\end{cases}    
\]

\subsection{线性算符与反线性算符}
\begin{itemize}
    \item 线性算符 \( \hat{A} \) \[\hat{A}\qty(c_1\ket*{\psi_1}+c_2\ket*{\psi_2})=c_1\hat{A}\ket*{\psi_1}+c_2\hat{A}\ket*{\psi_2}\]
    \item 反线性算符 \( \hat{B} \) \[\hat{B}\qty(c_1\ket*{\psi_1}+c_2\ket*{\psi_2})=c_1^*\hat{B}\ket*{\psi_1}+c_2^*\hat{B}\ket*{\psi_2}
        \]
\end{itemize}

\subsection{厄米共轭算符与厄米算符、反厄米算符}
\begin{itemize}
    \item 考虑 \( \ket*{\psi^\prime}=\hat{A}\ket*{\psi } \) 则 \( \bra{\psi^\prime}=\bra{\psi}\hat{A}^\dagger \)  \( \hat{A}^\dagger \) 是 \( \hat{A} \) 的厄米共轭算符 \[\begin{cases}
        \braket*{\psi^\prime}{\varphi}=\mel{\psi }{\hat{A}^\dagger}{\varphi}\\ 
        \braket*{\psi^\prime}{\varphi }=\braket*{\varphi}{\psi^\prime }^*=\mel{\varphi}{\hat{A}}{\psi}^*
    \end{cases}\Rightarrow\mel{\psi }{\hat{A}^\dagger}{\varphi}=\mel{\varphi}{\hat{A}}{\psi }^*\]
\end{itemize}
对于任意的 \( \ket*{\psi } \) 和 \( \ket*{\varphi} \) 若算符 \( \hat{B} \) 满足
\[
    \mel{\psi }{\hat{B}^\dagger}{\varphi}=\mel{\varphi}{\hat{A}}{\psi }^*
\]
则 \( \hat{B} \) 即为 \( \hat{A} \) 的厄米共轭算符 \( \hat{B}=\hat{A}^\dagger \) .

由上式关系可证:\[
    \begin{cases}
        \qty(\hat{A}\hat{B})^\dagger=\hat{B}^\dagger\hat{A}^\dagger\\ 
        \qty(\hat{A}^\dagger)^\dagger=\hat{A}\\ 
        \qty(\ketbra{\psi }{\varphi})^\dagger=\ketbra{\varphi }{\psi },\qty(\text{若}\ket*{\varphi}=\ket*{\psi}\Rightarrow\qty(\ketbra{\psi }{\psi })^\dagger=\ketbra{\psi }{\psi })\\ 
        c^\dagger=c^*,\qty(c\text{为复常数})
    \end{cases}\]

若 \( \hat{A}^\dagger =\hat{A}\) ,则称 \( \hat{A} \) 为厄米算符。(显然,投影算符 \( \hat{p}_\phi=\ketbra{\phi}{\phi } \) 是厄米算符)

厄米算符的若干特性
\begin{enumerate}
    \item  \( \hat{A}\ket*{a_i}=a_i\ket*{a_i} \) ,本征值 \( a_i \) 为实数
    \item 正交性:厄米算符的属于不同本征值的本征态正交 \( \braket*{a_i}{a_j}=\delta_{ij}=\begin{cases}
        1,i=j\\ 
        0,i\ne j
    \end{cases} \)  
    \item 完备性:  \( \ket*{\psi(t)}=\sum\limits_{i}^{}c_i(t)\ket*{a_i} \) 
\end{enumerate}

若 \( \hat{A}^\dagger=-\hat{A} \) ,则称 \( \hat{A} \) 为反厄米算符。

\subsection{幺正算符 \( \hat{U} \) (unitary operator)}

定义式: \( \hat{U}^\dagger=\hat{U}^{-1}\Rightarrow\hat{U}\hat{U}^{-1}=\hat{U}^{-1}\hat{U}=I\Rightarrow\hat{U}\hat{U}^\dagger=\hat{U}^\dagger\hat{U}=I \) ,其中 \( I \) 为单位矩阵。

 \( \hat{U}\ket*{\psi }\equiv\ket*{\psi^\prime},\hat{U}\ket*{\varphi}\equiv\ket*{\varphi^\prime} \) 

则有:
\[
    \braket*{\psi^\prime}{\varphi^\prime}=\mel{\psi }{\hat{U}^\dagger\hat{U}}{\varphi }=\braket*{\psi }{\varphi}
\]
因此幺正算符作用下,内积保持不变。

\textbf{幺正算符的本征值是模为 \( 1 \) 的复数}


\subsection{正规算符 \( \hat{N}  \) (normal operator) }

定义式: \( \qty[\hat{N},\hat{N}^\dagger]=0 \) 

厄米算符、幺正算符是正规算符。

正规算符属于不同本征值的不同本征态正交:
\[
    \hat{N}\ket*{n}=n\ket*{n},\hat{N}\ket*{m}=m\ket*{m}\Rightarrow\braket*{n}{m}=0\quad (n\ne m)
\]

谱展开
\[
    \hat{N}=\sum\limits_{n}^{}n\ketbra*{n}{n}
\]

\section{表现与表象变换}
\subsection[A 表象 A+ =A 分立谱]{ \( \hat{A}- \) 表象  \( \qty(\hat{A}^\dagger=\hat{A}) \) 分立谱}
\[
    \hat{A}\ket*{a_i}=a_i\ket*{a_i}
\]
\[
    \begin{cases}
        \braket*{a_i}{a_j}=\delta_{ij}&\text{正交性}\\ 
        \ket*{\psi(t)}=\sum\limits_{i=1}^{n}c_i(t)\ket*{a_i}&\text{完备性}  
    \end{cases}\]

完备性的算符表达式:
\[
    \begin{aligned}
        \ket*{\psi(t)}&=\sum\limits_{i=1}^{n}c_i(t)\ket*{a_i}\\ 
        \braket*{a_j}{\psi(t)}&=\sum\limits_{i=1}^{n}\braket*{a_j}{a_i}\\ 
        &=\sum\limits_{i=1}^{n}c_i(t)\delta_{ij}=c_j(t)\\ 
        c_i(t)&=\braket*{a_i}{\psi(t)}  \\ 
        \Rightarrow c_i(t)=\braket*{a_i}{\psi(t)}\\ 
        \Rightarrow \ket*{\psi(t)}&=\sum\limits_{i=1}^{n}\ket*{\psi(t)}\ket*{a_i}\\
        &=\sum\limits_{i=1}^{n}\ket*{a_i}\braket*{a_i}{\psi(t)}\\ 
        \Rightarrow I\ket*{\psi(t)}=\qty[\sum\limits_{i=1}^{n}\ketbra*{a_i}{a_i}]\ket*{\psi(t)}\\ 
        I&=\sum\limits_{i=1}^{n}\ketbra*{a_i}{a_i}
    \end{aligned}
\]
\begin{itemize}
    \item  \( \ket*{\psi(t)}=I\ket*{\psi(t)}=\sum\limits_{i=1}^{n}\ketbra*{a_i}{a_i}\ket*{\psi(t)} \) 
    \item  \( \ket*{b_j}=I\ket*{b_j}=\sum\limits_{i=1}^{n}\ketbra*{a_i}{a_i}\ket*{b_j} \)   
\end{itemize}
称采用了 \( \hat{A}- \) 表象。

而 \( \ket*{a_1},\cdots,\ket*{a_n} \) 称为该 \( n \) -维希尔伯特空间的基矢。
\begin{enumerate}
    \item  \( \hat{A}- \) 表象中的态矢量的表示 \[
        \ket*{\psi(t)}=\sum\limits_{i=1}^{n}\ketbra*{a_i}{a_i}\ket*{\psi(t)}
    \] 不同的态矢量对于不同的展开系数集合: \( \left.\{ \ketbra{a_i}{\psi(t)} \right.\} \) 
    \item  \( \ket*{\psi(t)}\Rightarrow\begin{pmatrix}
        \braket*{a_1}{\psi(t)}\\ 
        \vdots \\ 
        \braket*{a_n}{\psi(t)}
    \end{pmatrix}\equiv\psi(t) \)  \( \braket*{a_i}{\psi(t)} \) 为几率幅。
    \item  \( \bra{\psi(t)}=\bra{\psi(t)}I=\sum\limits_{i=1}^{n}\braket*{\psi(t)}{a_i}\bra{a_i}=\sum\limits_{i=1}^{n}\braket*{a_i}{\psi(t)}^*\bra{a_i} \) 
    \item  \( \bra{\psi(t)}=\qty(\braket*{a_1}{\psi(t)}^*,\cdots,\braket*{a_n}{\psi(t)}^*)=\equiv\psi^(t)(t) \) 
    \item 内积: \[\begin{aligned}
        \braket*{\varphi}{\psi}&=\mel**{\varphi}{I}{\psi}\\ 
        &=\sum\limits_{i=1}^{n}\braket*{\varphi}{a_i}\braket*{a_i}{\psi }\\ 
        &= \qty(\braket*{a_1}{\psi(t)}^*,\cdots,\braket*{a_n}{\psi(t)}^*)=\equiv\psi^(t)(t)  \begin{pmatrix}
            \braket*{a_1}{\psi(t)}\\ 
            \vdots \\ 
            \braket*{a_n}{\psi(t)}
        \end{pmatrix}\\ 
        &=\varphi^\dagger\psi 
    \end{aligned}
    \]
    \item  \( \hat{A}- \) 表象中的算符表示 \[
        \begin{aligned}
            \hat{F}&=\qty(\sum\limits_{i}^{}\ketbra{a_k}{a_i}\hat{F}\sum\limits_{j}^{}\ketbra*{a_j}{a_j})  \\ 
            & =\sum\limits_{jk}^{}\ketbra{a_i}{a_i}\hat{F}\ketbra{a_j}{a_j}\\ 
            & =\sum\limits_{jk}^{}\ket{a_i}F_{ij}\bra{a_j}
        \end{aligned}
    \]
    其中 \( F_{ij}=\mel{a_i}{F}{a_j} \) .
    \item 算符 \( \hat{F} \) 对应于一个 \( n\times n \) 的方阵。
    \[
        \hat{F}\longrightarrow\begin{pmatrix}
            F_{11} & F_{12} & \cdots & F_{1n} \\ 
            \vdots & \vdots & \ddots & \vdots \\ 
            F_{n1} & F_{n2} & \cdots & F_{nn} 
        \end{pmatrix}\]
\end{enumerate}


 \( \hat{F} \) 和 \( \hat{F}^\dagger \) 的矩阵元之间的关系
\[
\begin{aligned}
    F_{ij}&=\mel{a_i}{\hat{F}}{a_j}=\mel{a_j}{\hat{F}^\dagger}{a_i}^* \\ 
    &=\qty(F_{ji}^{\dagger})^*\\ 
    \rightarrow F_{ij}^{\dagger}=F_{ji}^{*}
\end{aligned}
\]

对于厄米算符: \( \hat{F}^\dagger=\hat{F},F_{ij}^{\dagger}=F_{ij} \Rightarrow F_{ij}=F_{jk}^{*} \) 

 \( \hat{U} \) 与 \( \hat{U}_{}^{-1} \) 的矩阵元之间的关系( \( \hat{U}^\dagger=\hat{U}_{}^{-1} \) )
 \[
     \begin{aligned}
         U_{ij}^{-1}=U_{ij}^{\dagger}=U_{ji}^{*}\\ 
         \Rightarrow U_{ij}^{-1}=U_{ji}^{*}     
     \end{aligned}
\]

算符 \( \hat{A} \) 的矩阵元( \( \hat{A}- \) 表象)
\[
    \begin{aligned}
        A_{ij}=\mel{a_i}{\hat{A}}{a_j}=a_j\braket*{a_i}{a_j}=a_j\delta_{ij}^{}\text{(对角矩阵)}
    \end{aligned}
\]
\[\scriptsize
A=\mqty(\dmat{a_1,\ddots,a_i,\ddots,a_j,\ddots,a_n})   
\]
\[
   \begin{aligned}
    \Rightarrow\hat{A}&=\sum\limits_{ij}^{}A_{ij}^{}\braket*{a_i}{a_j} \\ 
    &=\sum\limits_{ij}^{}a_i\delta_{ij}^{}\ketbra{a_i}{a_j}=\sum\limits_{i}^{}a_i\ketbra{a_i}{a_i}
   \end{aligned} 
\]

\[
    \ketbra{\alpha}{\beta}=\begin{pmatrix}
        \braket*{a_i}{\alpha}\\\vdots\\\braket*{a_n}{\alpha}
    \end{pmatrix}\begin{pmatrix}
        \braket*{\beta}{a_1}&\cdots&\braket*{\beta}{a_n}        
    \end{pmatrix}=\begin{pmatrix}
        \braket*{a_1}{\alpha}\braket*{\beta}{a_i}&\cdots&\braket*{a_1}{\alpha}\braket*{\beta}{a_n}\\ 
        \vdots&\ddots&\vdots\\ 
        \braket*{a_n}{\alpha}\braket*{\beta}{a_1}&\cdots&\braket*{a_n}{\alpha}\braket*{\beta}{a_n}
    \end{pmatrix}\]


\begin{comment}
	
\chapter{基本功能介绍}
\section{基本构成}

本模板有卷、章、节、小节、次小节构成。目录只显示到小节。

\section{无序列表的使用}

\begin{rpg-list}
	\item 原模板的配色很好看
	\item 原模板的图片很好看
	\item 原模板的盒子也不错
\end{rpg-list}

\section{彩色盒子的使用}

\begin{muse-obox}[简单的彩色盒子]
	这是一个简单的盒子,中括号里面放上盒子的名称。
\end{muse-obox}


\begin{muse-obox}[颜色可以改变的]{0.0,0.5,0.5}
	盒子虽然很简答,但是可以改变颜色,花括号里面是典型的 rgb 确定颜色的方法。不写的话就和上一个盒子一样用默认的颜色,填的话就变成你要的颜色。
\end{muse-obox}
	
\begin{muse-titlebox}{凸显标题的盒子}
		这个盒子是另一个样式,能够凸显标题。
\end{muse-titlebox}

\begin{muse-quotebox}
	这个盒子是设计来引用的,引用的放在盒子里面就更加突出了。
\end{muse-quotebox}
	
\begin{muse-pbox}{有背景图片的盒子}
	这是一个带背景图片的盒子。
	\tcblower
	这是盒子的设计就是从 tcolotbox 的说明文档上直接抄来的,挺好看的。
\end{muse-pbox}
\newpage
\section{在顶部和底部添加优美的图片}
\museart[t]{img/art-top}  %Figures/art-top
\museart[b]{img/art-bottom} 
使用了之后会把页眉页脚这些的位置改变哦,方括号里面只能填 t 或者 b 。t 自然是 top ,b 就是 bottom 。实际上加的图片是浮动体,可能出现的位置不是那么理想。你喜欢的话就用吧。
%\zhlipsum[1]
\newpage
\section{透明的图片}
\museart*[t]{img/art-top}
\museart*[b]{img/art-bottom}
如果觉得不想改变整个版面的边距,实际上可以加上*号,这样加入的图片位置就在当前页面了,不会乱跑了。这样做的话,图片是带透明度的。担心透明度不满意?自己的图片透明度不合适怎么办?
\newpage
\section{为图片选择你想要的透明度}
\museart*[t]{img/art-top}{0.7}
\museart*[b]{img/art-bottom}
实际上这个指令可以加第三个参数,如同示例的,0.7是图片的不透明度。你可以选择0-1之间的任意值。1就是显示原图的样子,0就是完全透明了。

\section{定理等环境}

定理、定义、命题、例子这四个环境已经定义好了。可以直接使用。

关于引用方面。我使用 clevercref 宏包来完成,因为它比较智能,很好用。所以一下的交叉引用都是用它。如果你还是喜欢普通的引用的话也还是可以用的。

\begin{muse-example}{例子盒子的简单使用}
	这是一个例子。在盒子内部自己引用的话可以用 \cref{exa:\thetcbcounter} 里面具体的代码方法。但是这个引用相当鸡肋。出了盒子就不能引用了。当然要有 label 才好了。如果你没写 label 的话,就像我这个例子里面一样,那么我自动给它绑定了一个,可以用\cref{exa:1.7.1}。但是如果你在前面添加章节的话,自动编的 label 就全部改变了,所以还是老老实实的自己加 label 吧。不要不添加,除非你认为不需要引用。
\end{muse-example}

\begin{muse-example}{记得被它们加上 label 哦}{exa:with-label}
	加上 label 之后就可以放心使用了。建议加上前缀 exa: ,不加也可以,但是这样容易区分,免得自己后面都把 label 混了。引用一下自己的 \cref{exa:with-label}。这样子就不怕 label 乱了。
\end{muse-example}

\begin{muse-theorem}{一个定理}{thm:nonsense}
	这里面根本不是一个真正的定理,这只是一个例子啊。建议 label 加上前缀 thm: 。
\end{muse-theorem}
\cref{thm:nonsense}是一个很有用的定理啊,你看它告诉你怎么使用定理环境了。

\begin{muse-definition}{喜欢耍流氓的人就是流氓}{def:rogue}
	我们定义喜欢耍流氓的人为流氓。
\end{muse-definition}
我认为\cref{def:rogue}的定义不妥当,耍流氓也可以是把流氓耍了,这个把流氓耍了的人不应该叫做流氓。

\begin{muse-proposition}{该命题是一个假命题}{pro:false}
	我是一个假的命题。
\end{muse-proposition}
\cref{pro:false}是一个假命题。我该信吗?

\begin{muse-cmd}[Camuse]{一个命令行}{cmd:how-to-use}
git clone https://github.com/CamuseCao/RPG-LaTeX-Template
cd RPG-Latex-Template
latexmk -xelatex
\end{muse-cmd}
\cref{cmd:how-to-use}向我们展示了这个项目的基本使用。

\begin{muse-listing}{结果与代码}{lis:easy-equation}
\begin{equation}
  \alpha + \beta = \omega 
\end{equation}
\end{muse-listing}


\begin{muse-listing}*{代码与结果}{lis:another-easy-equation}
\begin{equation}
  \alpha + \beta = \omega 
\end{equation}
\end{muse-listing}

\begin{muse-hlisting}{上下版本的}{lis:lis}
\begin{equation}
  \alpha + \beta = \omega 
\end{equation}
\end{muse-hlisting}

\begin{muse-hlisting}[listing only,]{只有代码}{lis:lis2}
\begin{equation}
  alpha + \beta = \omega 
\end{equation}
\end{muse-hlisting}
\end{comment}
\end{document}